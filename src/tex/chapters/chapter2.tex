\chapter{Literature Review}

Following the initial work done by Axelrod, there are many other papers that
have tried to tackle the PD and make their conclusions on cooperation in
both theoretical and real life behavior. In this chapter we review some of this
work done in the IPD competitions, in spatial and evolutionary game theory.

\section{The Prisoner's Dilemma}
The PD was originally formulated by Merril M. Flood and Melvin Dresher,
who were working on the Flood-Desher Experiment at the RAND cooperation in 1950.
% include a reference
Later in 1950, the mathematician Albert W. Tucker presented the first formal
representation of the PD, titled  A Two-Person Dilemma in a seminar at
Stanford University \parencite{Gass005}.

A description of the PD, found in \parencite{Li2011} is as follows:
There are two players that simultaneously have to decide to whether Cooperate (C)
or Defect (D) with each other, without exchanging information.

\begin{itemize}
  \item If both players choose to cooperate they will both receive a reward (R)
  \item If a player defects and the other cooperates then the defector receives
  a temptation payoff (T) and the cooperator a sucker payoff (S)
  \item If both players defect they will both receive a penalty (P)
\end{itemize}

% This should be in chapter 1 I think. Or perhaps some of the stuff that's in
% chapter 1 should move to here. Let's chat about the general structure of
% everything.

Fig. 1 illustrates the payoffs matrix.~\ref{fig:pd_payoff}.

\begin{figure}[h!]
\centering
\includegraphics[scale=0.45]{pd_payoff.jpg}
\caption{The payoff matrix for the Prisoners Dilemma \parencite{Li2011}}
\label{fig:pd_payoff}
\end{figure}

Taking into account the assumptions that both players are both rational human
beings and that there is no way of communication between them, it can be shown
that in the standard form of the PD a pure Nash Equilibrium exists when both
players defect.  So the outcome of any one round game will be Defection.
% You said something very similar to this in chapter 1

The Prisoner’s Dilemma can also be studied for any given number of
iterations(IPD), where players play each other repeatedly. An extra condition
% No space before punctuation but space after it.
for the payoffs in the IPD are : T \(>\) R \(>\) P \(>\) S and R \(>\) 1/2(T+S).

\section{Tournaments}

In 1980, Robert Axelrod held two computer tournaments based on the Iterated
Prisoner's Dilemma as described above.
% Have you described this alread? Do you mean chapter 1? If so reference chapter
% 1 (\ref{}) so that the author knows what you mean. But again: this is a bit
% disjointed and repetitive.
In the first tournament 14 different
strategies written by scientists from different fields competed each other on a
tournament of 200 turns per game and in a round robin topology. Axelrod also added
two more strategies: a Random one and a Twin.
% What do you mean by the Twin?

A strategy called Tit for Tat was announced the winner of the first tournament.
Tit for Tat is a deterministic strategy that will always cooperate in the first
round and afterwards it copies the opponents last move. Surprisingly in the
second tournament held by Axelrod \parencite{Axelrod1980a} where 64 strategies
competed and all they writes had full knowledge of what have happened in the
first tournament, Tit for Tat managed to get first place again.

As explained by Axelrod \parencite{Axelrod1980b}, Tit for Tat, a simple strategy was able to
win the rest of sophisticated and more complex strategies based on three specific
characteristics of the strategy:

\begin{itemize}
  \item Niceness:  A strategy is categorized as nice if it was not the
                    first to defect, or at least, it will not do this until
                    the last few moves.
  \item Forgiveness: The propensity to cooperate in the moves after the
                     opponent defected.
  \item Clarity: After opponents identified that they were playing Tit for Tat
                 choose to cooperate for the rest of the game.
\end{itemize}

The first tournaments were an innovation in combining computer modeling and Game
Theory and in providing insights in the behavior emerging from simple dynamics.
Moreover, Axelrod was the first to speak about niceness, forgiveness and gave an
illustration that cooperation can be a victorious and advantageous strategy.

Add more tournaments and criticisms on Axelrod's work.

\section{Spatial Structure Tournaments}

Nowak and May in 1992, were the first ones that conducted a PD tournament using
spatial structure \parencite{Nowak_&_May1992}. Their tournament was a simple and
purely deterministic spatial version of the PD, with no memories among the
players and no strategical elaboration, they players could  either always
cooperate or defect.
% ...with players having no memory of previous rounds and... Also: This sentence
% is long. It is ALWAYS better to break sentences up. Short and sharp.
Most of the work done by various other authors are only
% You mention other authors: reference them.
using these two basic strategies. These experiments with spatial tournament were
the perfect tool to identify the effect that a different topology than a round
robin would have on previous studies.
% Given my suggestion in chapter one about clearly defining a spatial
% tournament. Perhaps here recall that the round robin corresponds to a spatial
% tournament on the complete graph
Indeed, Nowak and May manage to show that
such structure has an important effect as it was shown that cooperation could
merge for the simple PD.

Similarly, in \parencite{Lindgren_&_Nordahl1994} players were allowed to have
memory and therefore added complex strategies to the tournament such as Tit for
Tat and  Anti Tit for Tat. This was followed  by the work of
\parencite{Brauchli_&_Killingback_&_Doebelis1999} which introduced even more
complex strategies.

Furthermore, in the first tournament the spatial structure is defined
on a 2D square lattice where players can only play their von Neuman or Moore neighbors.
% Same comment as for chapter 1
A lattice is defined as a graph, but
in non paper the author has defined a spatial structure as graph, expect from
\parencite{Meng&Xia_etc2015} where they defined it as a network.

Another interesting aspect of spatial IPD games is evolution. Nowak and May, in
both their papers used a deterministic approach, such as the player in a
neighborhood would mirror the move of the neighbor with the highest payoff of
all.  An interesting approach in evolute was given by
\parencite{Meng&Xia_etc2015} where they used a two lattices topology  and a
player would mimic a random player base on a function that consider the utility
of the player. Where each player’s utility will not only consider his/her own
payoff, but also incorporate the payoff of corresponding partner from the other
lattice.

\section{Evolution and Moran process}

\section{Axelrod Python Library}

The Axelrod library is an open source Python package that allows for
reproducible game theoretic research into the Iterated Prisoner's Dilemma.  For
many of the tournaments aforementioned the original source code is almost never
available and in no cases is the available code well-documented, easily modified
pr released with significant test suites. Due to that reproducing the results
has not been an easy task.

However, Axelrod library manages to provide such a resource, with facilities for
the design of new strategies and interactions between them, as well as
conducting tournaments and ecological simulations for populations of strategies.
Because is an open source library it makes it easy for us to contribute to it
make modifications needed for this dissertation.

% This section on the library can/will be rewritten. See comment in chapter 1
% about citation.
