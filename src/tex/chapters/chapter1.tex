\chapter{Introduction}

The Prisoner's Dilemma(PD) is a well known example in Game Theory and in recent
years has become the gold standard of understanding evolution of co-operative
behavior \parencite{Lorberbaum1994}.
% Reference not in bibliography.bib
That is due to the PD's nature.  In the
example of the Prisoner's Dilemma (PD) two criminals have been arrested and
interrogated, with no way of communicating, by the police.  They are given only
two choices, to either cooperate with each other or to defect.  By rational
selection the outcome of a single run PD will alway be to defect, even though
mutual cooperation would maximise their reward.

In 1981, much interest was earned by the PD due to the work done by Axelrod and
Hamilton, \parencite{Axelrod_&_Hamilton1981}. They proved that cooperation
behaviour can emerge from the PD if there is a chance that the players will meet
again.  This is referred to as the Iterated Prisoner's Dilemma (IPD). Various
experiments were held using the PD the year to comes, such as the second
experiment of Robert Axelrod \parencite{Axelrod1980b}, ...add a list...  Another
paper was by Nowak and May \parencite{Nowak_&_May1992} that conducted their own
tournament in 1992 using a different topology than Axelrod.
% big gap here: from 1981 and lots of work straight to Nowak's paper. Also
% strange to mention the 1981 paper and then then 1980b paper skipping over
% Axelrod's first paper in 1980 (the one with 14 strategies).
In their experiment
they decided to use a spatial topology and come across some unanticipated
results. They provided proof that cooperative behavior can emerge from a PD
tournament in certain topologies.
% The jump to spatial tournaments comes to fast. You're going to need a lot more
% literature and a bit more explanation: why spatial? What does the normal
% tournament not cover? Also: consider some network literature (not related to
% the IPD) just to discuss how these things are studied (one of the books I gave
% you will be a goo place to start for that).

In conclusion, Axelrod's work has set a seed in Game Theory and computer modeling
illustrating aspects of moral and political philosophy. How cooperation can be
evolutionarily advantageous has been a topic of focus ever since in various fields,
such as biology, sociology, ecology and psychology.

\section{Problem Description}

Axelrod's tournament as many others where conducted using a round robin
topology, were each player faces all other players. Even so, spatial topology
still has not been fully explored with only a small number of papers focusing on
this specific topology, such as \parencite{Nowak_&_May1993,
Brauchli_&_Killingback_&_Doebelis1999, Meng&Xia_etc2015,
Lindgren_&_Nordahl1994}. A goal of this dissertation is to understand the
current state of the art in spatial prisoner’s dilemma tournaments.

In all of those papers, their interpretation of spatial structure is a 2D square
% Note that we want you to have a more comprehensive list of such literature so
% this section will need to be adjusted.
lattice with each player interacting with only their neighbors in their von
Neuman or Moore's neighborhood.
% These two terms need to be referenced and defined.


Even so, a formal definition of what a spatial
structure is not given. Also evolution in all this papers varies from
deterministic approaches to more stochastic ones. We will be trying to define
what a spatial structure is and the evolution such topologies could use, such as
Mooran process.
% The above paragraph needs a lot more.
% - No mention of evolution up until now: what do you mean by that? See the
%   Szabo paper for how they do it (first describing evo in general sense and
%   then discussing it in spatia throughout the various possible configs).
%   You'll illustrate that by referencing particular papers.
% - The formal defitinion goes back to above: I would simply say something like,
%   these are generally defined on lattice which correspond to Graphs (include
%   some pictures showing this for the two types of neighbourhoods) and also
%   reference some of the papers that DO consider graphs, then conclude by
%   \begin{definition} your actual rigorous definition of a spatial tournament.

To be able to accomplish all this we will be using a python library, the Axelrod
% see https://github.com/Axelrod-Python/Axelrod/blob/master/CITATION.rst
library, that will allow the construction of reproducible tournaments for the
IPD using different strategies, topologies and evolution. Code written for the
purpose of this dissertation has been contributed to the library: see [you'll
put your PR urls here].

Finally, another aspect of the IPD tournament will be explored. That is teamwork, that
was introduced in the IPD competitions in 2004 by a team from the University of Southampton.
% The work by the Southampton team is quite different to what we have planned.
% Nevertheless this will be included and discussed (reference).
% This little sentence could perhaps go in to the previous section.

\section{Structure of Dissertation}
This dissertation is organized into 7 Chapters. Proceeding this introduction:
\begin{itemize}
  \item In Chapter 2 we review previous literature dedicated to the PD/IPD
        and tournaments that have been conducted, different topologies and evolution.
\end{itemize}
