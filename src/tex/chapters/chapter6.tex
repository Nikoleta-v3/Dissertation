\chapter{Conclusions and Future Work}
\label{chap:Six}

In this chapter the conclusions of the experiments held in
~\autoref{chap:Three} and in ~\autoref{chap:Four}, as well as the results
of the genetic algorithm for training the new strategy, in ~\autoref{chap:Five}
are presented. Furthermore, the potential areas of development are discussed.

\section{Conclusions}
\subsection{Affects of Topology}

In ~\autoref{chap:Three} and in ~\autoref{chap:Four}, the effects of the spatial
topology on the strategies of the Axelrod-Python library has been studied. This
has been implemented by producing data from various types of spatial tournaments.

In ~\autoref{chap:Three}, initially the source code needed to perform spatial tournaments
has been implemented by contributing to the Axelrod-Python library. There after,
an initial experiment was conducted using simple network topologies.
The results of the analysis showed that some of the high ranks strategies
of the IPD tournament conducted by the Axelrod-Python library using a round robin
topology, such as Raider and PSO Gambler made appearance on the highest ranked
strategies of the experiment. The performance in the initial experiments have been
measured using the wining ratio and the normalized average score. The rest of
the highest strategies have been different based on the networks used for the
experiment. The research, as to why these strategies have been scoring higher than
the rest return non significant findings. Thus, the results showed that the ranking
could not be predicted, based on the experiments data. The fact that topology
was affecting the performances was clear but the trend behind it was not necessarily
defined. Also, wining ratio and normalized score were characterized as poor measures
as there have been conflicts between their results.

In ~\autoref{chap:Four}, an new experiment have been implemented, this time using
more sophisticated networks. A new attempt to identify the topology's effects
was held. In this attempt an new measure of performance was introduced. The median
normalized ranked was used instead of the two previous. The only strategy that had
made an appearance once again between the high ranks has been the PSO Gambler and
finest strategy of all has been Gradual. Gradual is among the top strategies in
Axelrod library. Furthermore, an attempt to identify the reasons behind performance
was once again undergone.

The strategies this time have been classified based on their median cooperating
ratio throughly the experiment. The top ranked strategies have all a median cooperating
ratio higher than 0.50, indicating that cooperators
dominated the first spots. Even so, some of the lowest ranking strategies have
also been characterized a high cooperators. Thus, cooperating is not the only
factor which helped their performance. A further classification has also been
produced using the network's attributes. Five classes have been created using
the connectivity and clustering coefficients. A test to whether the normalized
score was significant different between these groups have been performed and it
returned that there is indeed statistically significant difference.

Building a regression model has been the immediate step to identify further factors.
The regression showed that apart from cooperating ratio, connectivity and the
average neighborhood score also affected the median ranking. Even so, the overall model
had a low \(R\) square values, thus the results have been non significant. By testing
the model to each of the strategies individual, two strategies shown that 80\% of
their variation can be explained by the model. Whether this could be possible
for the rest of the strategies is not clear. This could have been an effect of
the strategies game play or even of the environment they happened to participate in.

\subsection{New LookerUp Strategy}

Throughout the experiments that have been conducted the top ranking strategies
have seem to be different based on the experiment and the topology. In an attempt
of identifying an overall successful strategy for a spatial tournament of
a random topology was held in ~\autoref{chap:Five}. The are various networks
that could have been studied during this chapter but for keeping in line of
the approach chosen in \autoref{chap:Four}, the topology could have been any
random graph of the following networks: , and complete.

For achieving the goal that have been set for these chapter, the work of
M Jones and his LookerUp strategy was mimicked. The results of newly performed
evolutionary algorithm returned that a LookerUp strategy that could outperform
the median rank of Gradual was not able to found for the given number of generations
our algorithm managed to reach. Even so, by performing some extra algorithms for further
cases, such as each topology individually returned that a winning strategy for
a spatial tournament, under specific conditions could emerge.

\section{Limitations and Further Research}
\subsection{Limitations}

In thisdissertation most of the code for the spatial topology, large parts of the gentic algorithm
have to be implemented. Thus the time for runing the experiment has been descreaded.
Most of the tournaments themself needed a lot time. Thus the the biggest constraing
of this dissertation has been time.

With futher time, the experiments in both chaoter could have been used for
more number of player , tournaments sizes, repitions and networks. This would
made possible for more data to be conducted and have more to run our analysis
wiht a hope of better results.

Data are evrything in the analysis. For exaple in hapter 3 only two specific
tournament sizes have been used. ANd in chapter4 now all method reach maximum capacity.

Futhermore more, for yhe genetic algorithm, time had been even less. The genetic
alorgitm and the spaital tournaments made it impossible to run big ames, with
manu players. more turns and repitions and more generations. Thus a better solutions
as in the futher reaserach could possible be achieved

- Time constrains.
- Generate more data
- Run more generations for genetic algorithm
\subsubsection{Further Research}

for further research, i would like to run more comple networks. even from the complex
networks ionyl three types have been shoosen were thousand of graphs exists.

THe regression models did not managed to explain any of the results. This could have been
because of the lack of data. Even s, more regresiion could be perform such as
forward elimination regression and logistic regression.

- More complex networks
- Regressions, forward regression and logistic regression
- Make LookerUp compete in experiments
