\chapter{Conclusions and Future Work}
\label{chap:Six}

In this chapter the conclusions of the experiments held in
~\autoref{chap:Three} and in ~\autoref{chap:Four}, as well as the results
of the genetic algorithm for training the new strategy, in ~\autoref{chap:Five}
are presented. Furthermore, the potential areas of development are discussed.

\section{Conclusions}
\subsection{Affects of Topology}

In ~\autoref{chap:Three} and in ~\autoref{chap:Four}, the effects of the spatial
topology on the strategies of the Axelrod-Python library has been studied. This
has been implemented by producing data from various types of spatial tournaments.

In ~\autoref{chap:Three}, initially the source code needed to perform spatial tournaments
has been implemented by contributing to the Axelrod-Python library. There after,
an initial experiment was conducted using simple network topologies.
The results of the analysis showed that some of the high ranks strategies
of the IPD tournament conducted by the Axelrod-Python library using a round robin
topology, such as Raider and PSO Gambler made appearance on the highest ranked
strategies of the experiment. The performance in the initial experiments have been
measured using the wining ratio and the normalized average score. The rest of
the highest strategies have been different based on the networks used for the
experiment.The research, as to why these strategies have been scoring higher than
the rest return non significant findings. Thus the results showed that the ranking
could not be predicted, based on the experiments data. The fact that topology
was affecting the performances was clear but not the trend behind it was not necessarily
defined. Also, wining ratio and normalized score were characterized as poor measures
as there have been conflicts between the results.

In ~\autoref{chap:Four}, an new experiment have been implemented, this time using
more sophisticated networks. A new attempt to identify the topology's effects
was held. In this attempt an new measure of performance was introduced. The median
normalized ranked was used instead of the two previous. The only strategy that had
made an appearance once again between the high ranks has been the PSO Gambler.
 

attempt to indentify the reasons behind performance
was once again undergone. The median normalized rank was indentified to be a much better
measure, but even so the results of the experiment have been once again random.
PSO Gambler did achieve a high place on again. But the regression showed that
only a few strategies could be predicted. And the rest seems to be affected
by different predictors. This could be an effect of the game play of each strategy
of even by the enviroment itself.

Thus further reserach is needed for validate resutls
\subsection{New LookerUp Strategy}

In an attempt to indentify a winning streatygy in the spaital random experiment
that have been conducted. The work of M Jones and his LookerUp strategy were mimiced.
A new strategy using an objective function based on the median normalized score
has been perforfm. Due to time costrainf the generations had to be stopped
quick. Thus this were the resutls

\section{Limitations and Further Research}
\subsection{Limitations}

In thisdissertation most of the code for the spatial topology, large parts of the gentic algorithm
have to be implemented. Thus the time for runing the experiment has been descreaded.
Most of the tournaments themself needed a lot time. Thus the the biggest constraing
of this dissertation has been time.

With futher time, the experiments in both chaoter could have been used for
more number of player , tournaments sizes, repitions and networks. This would
made possible for more data to be conducted and have more to run our analysis
wiht a hope of better results.

Data are evrything in the analysis. For exaple in hapter 3 only two specific
tournament sizes have been used. ANd in chapter4 now all method reach maximum capacity.

Futhermore more, for yhe genetic algorithm, time had been even less. The genetic
alorgitm and the spaital tournaments made it impossible to run big ames, with
manu players. more turns and repitions and more generations. Thus a better solutions
as in the futher reaserach could possible be achieved

- Time constrains.
- Generate more data
- Run more generations for genetic algorithm
\subsubsection{Further Research}

for further research, i would like to run more comple networks. even from the complex
networks ionyl three types have been shoosen were thousand of graphs exists.

THe regression models did not managed to explain any of the results. This could have been
because of the lack of data. Even s, more regresiion could be perform such as
forward elimination regression and logistic regression.

- More complex networks
- Regressions, forward regression and logistic regression
- Make LookerUp compete in experiments
