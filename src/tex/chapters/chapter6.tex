\chapter{Conclusions and Future Work}
\label{chap:Six}

In this chapter the conclusions of the experiments held in
~\autoref{chap:Three} and in ~\autoref{chap:Four}, as well as the results
of the genetic algorithm for training the new strategy, in ~\autoref{chap:Five}
are presented. Furthermore, the potential areas of development are discussed.

\section{Conclusions}
\subsection{Affects of Topology}

In ~\autoref{chap:Three} and in ~\autoref{chap:Four}, the effects of the spatial
topology on the strategies of the Axelrod-Python library has been studied. This
has been implemented by producing data from various types of spatial tournaments.

In ~\autoref{chap:Three}, initially the source code needed to perform spatial tournaments
has been implemented by contributing to the Axelrod-Python library. There after,
an initial experiment was conducted using simple network topologies.
The results of the analysis showed that some of the high ranks strategies
of the IPD tournament conducted by the Axelrod-Python library using a round robin
topology, such as Raider and PSO Gambler made appearance on the highest ranked
strategies of the experiment. The performance in the initial experiments have been
measured using the wining ratio and the normalized average score. The rest of
the highest strategies have been different based on the networks used for the
experiment. The research, as to why these strategies have been scoring higher than
the rest return non significant findings. Thus, the results showed that the ranking
could not be predicted, based on the experiments data. The fact that topology
was affecting the performances was clear but the trend behind it was not necessarily
defined. Also, wining ratio and normalized score were characterized as poor measures
as there have been conflicts between their results.

In ~\autoref{chap:Four}, an new experiment have been implemented, this time using
more sophisticated networks. A new attempt to identify the topology's effects
was held. In this attempt an new measure of performance was introduced. The median
normalized ranked was used instead of the two previous. The only strategy that had
made an appearance once again between the high ranks has been the PSO Gambler and
finest strategy of all has been Gradual. Gradual is among the top strategies in
Axelrod library. Furthermore, an attempt to identify the reasons behind performance
was once again undergone.

The strategies this time have been classified based on their median cooperating
ratio throughly the experiment. The top ranked strategies have all a median cooperating
ratio higher than 0.50, indicating that cooperators
dominated the first spots. Even so, some of the lowest ranking strategies have
also been characterized a high cooperators. Thus, cooperating is not the only
factor which helped their performance. A further classification has also been
produced using the network's attributes. Five classes have been created using
the connectivity and clustering coefficients. A test to whether the normalized
score was significant different between these groups have been performed and it
returned that there is indeed statistically significant difference.

Building a regression model has been the immediate step to identify further factors.
The regression showed that apart from cooperating ratio, connectivity and the
average neighborhood score also affected the median ranking. Even so, the overall model
had a low \(R\) square values, thus the results have been non significant. By testing
the model to each of the strategies individual, two strategies shown that 80\% of
their variation can be explained by the model. Whether this could be possible
for the rest of the strategies is not clear. This could have been an effect of
the strategies game play or even of the environment they happened to participate in.

\subsection{New LookerUp Strategy}

Throughout the experiments that have been conducted the top ranking strategies
have seem to be different based on the experiment and the topology. In an attempt
of identifying an overall successful strategy for a spatial tournament of
a random topology was held in ~\autoref{chap:Five}. The are various networks
that could have been studied during this chapter, but for keeping in line with
the approach of \autoref{chap:Four}, the topology was a random selection between
the following networks: Watts Strogatz, Erd\"\{o\}s R\'\{e\}nyi and complete networks.

For achieving the goal that have been set for this chapter, the work of
M Jones and his LookerUp strategy was mimicked. The results of newly performed
evolutionary algorithm returned that a LookerUp strategy that could outperform
the median rank of Gradual was not able to be found for the given number of generations
our algorithm managed to run. Even so, by performing some extra algorithms for further
cases, such as for each topology individually, returned that a winning strategy for
a spatial tournament under specific conditions could emerge.

\section{Limitations and Further Research}
\subsection{Limitations}
\label{sub:limitations}
For this dissertation a total of three experiments have been carried out and
each of the experiments included separately stages. Firstly, the source code of implementing
each had to be written. Secondly, running each experiment for producing required data.
For the data to be created a large number of spatial
tournaments and an evolutionary algorithm had to be performed. Finally, an
analysis had to held for each outcome. All these three steps individually required
an amount of time and computational resources, thus the biggest constrain for
this dissertation has been time.

Due to this constrain a lot of the parameters, such as tournaments turn, repetitions,
tournament's size etc. had to be set to minimum. For example, in \autoref{chap:Three},
the tournament size has been deterministically chosen. In \autoref{chap:Four}
not all methods achieved their maximum capabilities and in \autoref{chap:Five} the
genetic algorithm had a time limit of 70 hours.

\subsubsection{Further Research}

Due the limitations as described in \autoref{sub:limitations}, some further
research ideas include the reproduction of these experiments by setting the
parameters to maximum. For example the number of repetitions could be set to 100
and the tournament size could range for 2 to 132. This holds for the evolutionary
algorithm as well. Where the algorithm could be performed for a higher time limit.

Moreover, there are various of types of complex networks but studying them all
is impossible. Though, a new experiment could be conducted where the topology
could be randomly selected from a pool of greater than 3 complex networks. The
regression models in this dissertation did not manage to explain any of the results.
This could have been because of the lack of data. Even so, further regression
analysis could have been perform. Both forward elimination model and a logistic
regression could be exceptional choices.

Furthermore, the IPD tournament itself included various types and concepts that
have been studied throughly the years. Some of the have been briefly discussed
in the initial chapters. Concepts such as noise, probabilistic endings and evolutionary
tournaments. Some further ideas include the implementation of these types in the
spatial tournaments. A spatial tournament could easily include noise, thus altering
the actions of some strategies randomly. A spatial probabilistic ending tournament,
would mean a tournament where for where the length of each tournament is not
constant for all encounters: after each turn the tournament ends with a given
probability. Finally, an spatial evolutionary tournament have been studied before,
and it has been the origin approach followed by Nowak in 1992, where strategies
would alter their game play based on the winner of their generation. An evolutionary
tournament could shred some light to various questions. Such as: would the
tournaments achieve a ESS and whether cooperation or defection would emerge as
the dominant action. 
