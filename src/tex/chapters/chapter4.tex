\chapter{Complex Networks Experiments}
\label{chap:Four}

\section{Introduction}
In this chapter experiments with more complex networks have been performed.
An introduction to these networks and the reasons behind their selection has been
covered in the begging of the chapter. The experiments structure, for each topology,
have also been analyzed. Finally, an analysis on the data produced for these
experiments, has been performed. The winning ratio as well as the normalized
average score have been examined.

The issues raised in~\ref{ch:Three}, have been address in the current chapter,
altering various approaches. For example, a measure of similarity between the
strategies has been used. The measure used, is based on the average cooperating
ratio of the strategies. Furthermore, median ranking of each strategy has been
looked into, for it is believed to be a more appropriate measure of performance.
Compared to winning ratio. Regression models for winning ratio, average score
and median rank have been build. Furthermore, round robin in treated as
a spatial tournament of a complete topology.

\subsection{Networks}

A complex network is a graph with non-trivial topological features;
features that can not occur in networks such as the one studied in ~\ref{}.
Such features include, a heavy tail in the degree distribution, a high
clustering coefficient, and hierarchical structure. Complex networks are commonly
used to study computer network, technological networks and social networks.

They will also be used in the experiment conducted in this dissertation. Small
world and random graphs, and more specifically, the following networks are
studied :

\begin{itemize}
  \item Watts-Strogatz, a model for generating graphs with small-world properties
  \item Erdős–Rényi, a model for generating random graphs
  \item Complete Graph, a graph with the properties of a round robin topology
\end{itemize}

Each of the experiment has a different set of parameters, thus the size of
data produced by each experiment is different. The structure of each experiment
is analyzed on the following subsection.

\subsection{Experiments Structure}
\subsubsection{Small-world}

Watts-Strogatz, is the graph generator model used in this experiment. Networkx,
provide us with a function for such graphs. The documentation for the
Watts-Strogatz function, can be found here
\url{https://networkx.github.io/documentation/development/reference/generated/networkx.generators.random_graphs.watts_strogatz_graph.html#networkx.generators.random_graphs.watts_strogatz_graph}
, and it takes \(n\), \(k\) and \(p\) as arguments. Where \(n\) is the number
of nodes, \(k\) is nearest neighbors each node is connected to, and \(p\) the
he probability of rewiring each edge. Moreover, the tournament size is not
going to be deterministic. The tournament size will range from 2 to 50.
Same will be done of the initial size, it will range from 2 to tournament size
minus one. The chosen players are also shuffled and re placed on the graph.
A pseudo code illustrating the structure of the experiment is shown in~\ref{small-world-experiment}:

\begin{algorithm}
  \caption{Small-world Experiment}\label{small-world-experiment}
  \begin{algorithmic}
  \Procedure{Small World}{}
  \BState \emph{loop}:
  \For {$\textit{tournament_size} \gets \textit{2 to 50}$}
  \For {$\textit{initial_neighborhood_size} \gets \textit{1 to tournament_size-1}$}
  \For {$\textit{i} \gets \textit{ 0 to 100}$}
  \State $player \gets \textit{random.strategies}$.
  \For {$p \gets \textit{0 to 10}$}
  \State $G \gets \textit{create.watts.strogatz.graph}$.
  \State $edges \gets \textit{G.egdes}$
  \State $results \gets \textit{play.tournament}$.
  \emph{loop}.
  \EndFor
  \EndFor
  \EndFor
  \EndFor
  \EndProcedure
  \end{algorithmic}
\end{algorithm}

\subsubsection{Random}

For recreating a random network experiment, Erdős–Rényi has been used.
Again networkx provide a function to easily generate such graph\url{https://networkx.github.io/documentation/development/reference/generated/networkx.generators.random_graphs.binomial_graph.html#networkx.generators.random_graphs.binomial_graph}.
The arguments taken but the function are the following : \(n\) as the number
of nodes and \(p\) as the probability for edge creation. Again, the tournament
size is not going to be deterministic. The tournament size will range from 2 to 50.
The chosen players are also shuffled and re placed on the graph.
A pseudo code illustrating the structure of the experiment is shown in~\ref{small-world-experiment}:

\begin{algorithm}
  \caption{Random Experiment}\label{simple-experiment-rules}
  \begin{algorithmic}
  \Procedure{Random}{}
  \BState \emph{loop}:
  \For {$\textit{initial_neighborhood_size} \gets \textit{1 to tournament_size-1}$}
  \For {$\textit{i} \gets \textit{ 0 to 100}$}
  \State $player \gets \textit{random.strategies}$.
  \For {$p \gets \textit{0 to 10}$}
  \State $G \gets \textit{create.watts.strogatz.graph}$.
  \State $edges \gets \textit{G.egdes}$
  \State $results \gets \textit{play.tournament}$.
  \emph{loop}.
  \EndFor
  \EndFor
  \EndFor
  \EndProcedure
  \end{algorithmic}
\end{algorithm}

\subsubsection{Complete}
\begin{algorithm}
  \caption{Complex Experiments Rules}\label{simple-experiment-rules}
  \begin{algorithmic}
  \Procedure{Complete}{}
  \BState \emph{loop}:
  \For {$\textit{initial_neighborhood_size} \gets \textit{1 to tournament_size-1}$}
  \For {$\textit{i} \gets \textit{ 0 to 100}$}
  \State $player \gets \textit{random.strategies}$.
  \For {$p \gets \textit{0 to 10}$}
  \State $G \gets \textit{create.watts.strogatz.graph}$.
  \State $edges \gets \textit{G.egdes}$
  \State $results \gets \textit{play.tournament}$.
  \emph{loop}.
  \EndFor
  \EndFor
  \EndFor
  \EndProcedure
  \end{algorithmic}
\end{algorithm}


\setion{Analysis}

\subsection{Network Analysis}

\subsection{Preliminary Analysis}

\subsection{Classification}

\subsection{Winning Ratio}

\subsection{Median Rank}

\subsection{Normalized Average Score}

\subsection{Regression}

\section{Summary}
