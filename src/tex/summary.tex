\chapter{Summary}

Ever since 1980, when a simple Iterated Prisoner's Dilemma(IPD) computer
tournament was conducted with 13 participants, new IPD tournaments have been run
and new strategies are constantly developed. This has inspired scientists to develop
an open source Python library, for reproducing all the research done on the topic
of the IPD. The library is called Axelrod-Python library, and furthermore it continuously holds
an IPD tournament with the, currently 132, implemented strategies.

Axelrod-Python library, soon became a powerful tool for the Game Theory society.
Even so, the library was lacking the ability of creating tournaments of a different
topology of that of a round robin one. During this dissertation, the ability to
produce tournaments of a network topology; spatial tournament, was materialized.
Instantly, new questions were raised;such as what are the effects of the topology?

For addressing this questions the following approach was chosen. Identify the
effects of the topology on the performance of the strategies of the library and
does any of this strategies perform well in different tournaments of different
topologies. Thus throughly the research experiments have been conducted. In these
experiments simple and complex networks respectively have been used as topologies
to generate numerous of spatial tournaments. Furthermore, non of the
Axelrod-Python library strategies was characterized as an overall satisfactory
strategy through the experiments. For this reason an new strategy has been trained
using an evolutionary approach.
