\chapter{Summary}
This dissertation investigates, the aspect of network topology, for the popular
Iterated Prisoner's Dilemma tournaments, a particular topic of  Game
Theory, and it is performed by taking advantage of the Axelrod Python Library.

This dissertation was complete during the taught program of Operational Research
and Applies Statistics of Cardiff University, under the supervision of Dr Vince
Knight.

In 1980s, a tournament for the game of Iterated Prisoners Dilemma, spawned to
life an new area of research for Game Theory. Various scientist, from different
research fields, have every since contributed in this research with concepts
of noise, altered set of rules, probabilistic endings, further strategies and etc.
In 1992, a new topology was introduced, that of a spatial structure. In this
new topology, players will not interact will all the player, but instead were
allocated into networks and would interact with players to whom they have a
connection. This dissertation has been focused on understanding the aspects of
this new topology, by using network analysis.

Two experiments have been conducted. The first experiment, has been an initial
experiment trying to understand the spatial structure and find proper measures
of performance accuracy. Simple networks have been used for the tournament
topologies. Particularly, there have been three types of tournaments.
The first, a spatial tournament with a periodic lattice topology, the
second a spatial tournament of a cyclic topology and a round robin tournament.

The second experiment, was held to understand in depth the affects of the
topology. Using three different type of methods, spatial tournament have been
played. and afterwards their results have been studied. The median normalized
rank have been chosen as the performance measure.

Even so, out of all 132 strategies that have been given to us from the Alexrod- Python
library, none managed to outperform the rest. Thus in the final chapter, a
strategy has been trained,

From the mathematical aspects Graph Theory, Machine Learning, Clustering and
Genetic algorithm have been explored, to answer to the questions that this
dissertation have set.
