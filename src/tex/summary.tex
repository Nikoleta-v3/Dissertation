\chapter{Summary}

Ever since 1980, when a simple Iterated Prisoner's Dilemma (IPD) computer
tournament was conducted with 13 participants, new IPD tournaments have been run
and new strategies are constantly developed. This has inspired scientists to develop
an open source Python library, for reproducing all the research done on the topic
of the IPD. The library is called Axelrod-Python library, and furthermore it continuously holds
an IPD tournament with the, currently 132, implemented strategies.

The Axelrod-Python library, has become a powerful tool for the Game theoretic community.
Even so, up until this dissertation it lacked the ability to create tournaments on a variety of
topologies (a part from the basic round robin).
During this dissertation, the ability to
produce tournaments of a network topology; spatial tournament, was materialized.
Instantly, new questions were raised;such as what are the effects of the topology?

For addressing this question the following approach was chosen. Identify the
effects of the topology on the performance of the strategies of the library and
does any of this strategies perform well in different tournaments of different
topologies. Thus throughly the research experiments have been conducted. In these
experiments simple and complex networks respectively have been used as topologies
to generate a number of spatial tournaments. Furthermore, non of the
Axelrod-Python library strategies was characterized as an overall satisfactory
strategy through the experiments. For this reason a variety of new strategies have
been trained using a genetic algorithm.
