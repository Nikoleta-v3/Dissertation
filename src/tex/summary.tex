\chapter{Summary}

Since 1984, when a simple Iterated Prisoner's Dilemma(IPD) computer tournament was
conducted with 16 participants, various IPD tournaments have been and new strategies
are constantly developed. This has inspired scientists to create a new Python
library, for reproducing all the various works done on the topic of the IPD. This
library is called Axelrod-Python library and further it continuously holds
an IPD tournament with the strategy that have been implemented(currently 132).

Axelrod-Python library, soon became a powerful tool for the Game Theory society.
Even so, the library was lacking the ability of creating tournaments of various
topologies apart of a round robin one. During this dissertation, this was implemented
by contributing to the library's source code. Specifically, tournaments will a
network topologies: spatial tournaments. Instantly, new questions were raised.
How does the topology of the tournament affect the performance of the existed
strategies of the library and does any of this strategies perform well in
different tournaments of different topologies ?

To answer this questions throughly, experiments have been conducted. In these
experiments simple and complex networks respectively have been used as topologies
to generate numerous of spatial tournaments. This was done to identify the effects
of the topologies on the 132 strategies of the library. Furthermore, non of 132
Axelrod-Python library strategies was characterized as an overall satisfactory
strategy thought the experiments. For this reason an new strategy has been trained
using an evolutionary approach.
