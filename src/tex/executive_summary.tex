\chapter{Executive Summary}
This dissertation investigates, the aspect of network topology, for the popular
Iterated Prisoner's Dilemma tournaments, a particular topic of  Game
Theory, and it is performed by taking advantage of the Axelrod-Python library.
From the mathematical aspects, this dissertation covers, Graph Theory,
Machine Learning, Clustering and Genetic algorithms.

This dissertation was completed during the taught program of Operational Research
and Applies Statistics of Cardiff University, under the supervision of Dr Vince
Knight.

In the 1980s, a tournament for the game of Iterated Prisoners Dilemma, spawned to
life a new area of research for Game Theory. Various scientists, from different
research fields, have ever since contributed to this research with concepts
such as noise, altered set of rules, probabilistic endings, further strategies and etc.
In 1992 new aspect was introduced: tournaments on different topologies.
In this new topology, players will not interact will all the player, but instead
were allocated on to networks and would interact with all the players to whom they have a
connection. This dissertation has focused on understanding the aspects of
this new topology, by using network analysis.

The research was made possible due a particular source Python package, the
Axelrod- Python library. The library was developed specifically for
reproducing the research into the Iterated Prisoner's Dilemma. The implementation
of spatial tournaments has a been an addition to the capabilities of the library,
which was materialized during this dissertation. % I have no idea if this makes sense by it sounds sophisticated *^*

Throughout this work there are two underlying strands: data generation (carried
out by running a large number of tournaments) and data analysis.
Two experiments have been conducted. The first experiment, has been an initial
experiment trying to understand the spatial structure and find proper measures
of performance accuracy. Simple networks have been used for the tournament
topologies. Particularly, there have been three types of tournaments.
The first, a spatial tournament with a periodic lattice topology, the
second a spatial tournament of a cyclic topology and a round robin tournament.

The second experiment, was held to understand in depth the affects of the
topology. Complex networks have been used as the tournaments topologies for the
reason that they are a more realistic illustrations of real life systems.
Specifically, the data have been produced using three different methods for
generating a large number of complex networks. These networks could have small world,
random or complete network properties. For each network a spatial tournament has
been played and the results of those tournaments have been analyzed. Their analysis
was aimed to identify the affects of the topologies and possibly classify whether
a strategy of the Axelrod-Python library could have an overall satisfactory performance
thought out all this spatial tournaments.

Considering the results of the analysis held, an attempt to create a new strategy
is held in the following chapter. This new strategy has origins of an already existence
strategy in the Alexrod-Python library. Compared to the original version the new
strategy has been trained, using a genetic algorithm, to aim for a satisfactory
performance in the spatial Iterated Prisoner's Dilemma.

From the research conducted to this point the following conclusions are made.
The spatial topology has effects on the strategies, as the results were different
from those of a round robin topology, and topology has a different affect. An attempt
to predict the performance of the strategies return no significant results. Though
it is believed by producing further data with various additional complex networks
more validate points could emerge. Furthermore, the research of the new strategy
returned that for specific topologies an accomplished strategy exist and has been
verified. Nonetheless, that did not hold for all the examined topologies. Thus,
with more time at hands the sophisticated look up for such strategies would be
continued.

Finally, the topic of the IPD included various other topics and type of tournaments.
Such as noisy and probabilistic endings tournaments. Some further ideas include
experiments with various of these types of tournaments combined.
