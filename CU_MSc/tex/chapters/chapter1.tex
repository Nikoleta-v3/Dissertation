\chapter{Introduction}

The Prisoner's Dilemma(PD) is a well known example in Game Theory and in recent
years has become the gold standard of understanding evolution of
co-operative behavior \parencite{Lorberbaum1994}. That is due to the PD's nature.
In the example of the Prisoner's Dilemma (PD) two criminals have been arrested and
interrogated, with no ways of communicating,by the police.
They are given only two choices, to either cooperate with each other or defect.
By rational selection the outcome of a single run PD will alway be to defect,even
if by cooperating they can achieve the maximum reward.

In 1981, much interest was earned by the PD due to the work done by Axelrod and
Hamilton,\parencite{Axelrod_&_Hamilton1981}, who proved that cooperation dehaviour
can emerge from PD if there is probability that the players will meet again. The
Iterated Prisoner's Dilemma (IPD).Various experiments were held using the PD the
year to comes,such as the second experiment of Robert Axelrod \parencite{Axelrod1980b},
...add a list...
Another paper was by Nowak and May \parencite{Nowak_&_May1992} that conducted
their own tournament in 1992 using a different topology than Axelrod. In their
experiment they decided to use a spatial topology and come across some not
anticipated results. They provided proof that cooperational behavior can merge
from a PD tournament if using the right topology.

In conclusion, Axelrod's work has set a seed in Game Theory and computer modeling
illustrating aspects of moral and political philosophy. How cooperation can be
evolutionarily advantageous has been a topic of focus ever since from various fields,
such as biology, sociology, ecology and psychology.\parencite{Nowak_&_May1992}.
Trying to prove that cooperation can be evolutionarily advantageous.

\section{Problem Description}
Axelrod's tournament as many others where conducted using a round robin topology,
were each player faces all other players. Even so, spatial topology still has
not been fully explored with only a small number of papers focusing on this
specific topology, such as \parencite{Nowak_&_May1993}, \parencite{Brauchli_&_Killingback_&_Doebelis1999},
\parencite{Meng&Xia_etc2015},\parencite{Lindgren_&_Nordahl1994}. A goal of this
dissertation is understand the current state of the art in spatial prisoner’s dilemma
tournaments.

In all these papers, their interpretation of spatial structure is a 2D square lattice with
each player interacting with only their neighbors in their von Neuman or Moore's
neighborhood. Even so, a formal definition of what a spatial
structure is not given. Also evolution in all this papers varies from deterministic
approaches to more stochastic ones. We will be trying to define what a spatial structure
is and the evolution such topologies could use, such as Mooran process.

To be able to accomplish all this we will be using a python library, Axelrod library,
that will allow me to reproduce tournaments for the IPD using different strategies,
topologies and evolution. By contributing to the Axelrod library we will be able
to reproduce some of the tournaments aforementioned and look in more depth in spatial
IPD tournaments.

Finally, another aspect of the IPD tournament will be explored. That is teamwork, that
was introduced in the IPD competitions in 2004 by a team from the University of Southampton.

\section{Structure of Dissertation}
This dissertation is organized into 7 Chapters. Proceeding this introduction:
\begin{itemize}
  \item In Chapter 2 we review previous literature dedicated to the PD/IPD
        and tournaments that have been conducted, different topologies and evolution.
\end{itemize}
